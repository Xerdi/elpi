\documentclass{ltxdoc}
\usepackage[titlepage]{gitinfo-lua}
\usepackage{listings}
\usepackage{elpi}
\usepackage{multicol}
\usepackage{calc}
\usepackage{hyperref,embedfile}
\usepackage{textcomp}
%\usepackage{navigator}

% embedding links:
% https://tex.stackexchange.com/questions/19025/embed-link-to-an-embedded-file
% https://stackoverflow.com/questions/59868631/can-i-attach-embed-a-file-to-a-latex-document-and-hyperlink-it-to-a-word-for-sav
\embedfile[desc={Parameter Specification Example},filespec=example-specification.yaml]{example/example-specification.yaml}
\embedfile[desc={Values Specification Example},filespec=example.yaml]{example/example.yaml}

%\embeddedfile[Parameter Specification Example]{example-spec}[example-specification.yaml]{example/example-specification.yaml}
%\embeddedfile[Values Specification Example]{example}[example.yaml]{example/example.yaml}

\def\projecturl{https://github.com/Xerdi/elpi}

\newcommand\showexample[5][15pt]{%
\begin{minipage}[t]{.5\linewidth - .5 \columnsep}%
\lstinputlisting[firstnumber={#2},linerange={#3},style=YAML,frame=single,numbers=left,xleftmargin=0pt,numbersep=10pt]{example/example-specification.yaml}
\end{minipage}\hspace*{\columnsep}%
\begin{minipage}[t]{.5\linewidth - .5 \columnsep}%
\lstinputlisting[firstnumber={#4},linerange={#5},style=YAML,frame=single,numbers=left,xleftmargin={#1},numbersep=10pt]{example/example.yaml}
\end{minipage}\\%
}

\begin{document}
    \title{Extended \LaTeX{} Parameter Interface\thanks{This document corresponds to \texttt{doc-payload-spec} version \gitversion, written on \gitdate}}
    \maketitle

    \begin{abstract}
        This package is meant for interfacing Lua\LaTeX{} documents in a more programmatically way with YAML\@.
        The payload specification is a set of document parameters, which can be specified using \TeX{} macros or using an external specification file called a \textit{recipe}.
        The payload itself is always provided as external file and can optionally be passed through the commandline to avoid changes in the \LaTeX{} document source.
    \end{abstract}

    \section{Usage}
    Before loading any \meta{payload}, the \meta{recipe} has to be loaded or specified in advance.
    Otherwise, errors will occur.
    \DescribeMacro{\loadrecipe}
    In order to load a recipe the macro \cmd{\loadrecipe}\oarg{namespace}\marg{filename} can be used.
    Where the \meta{filename} is a YAML file with its corresponding extension.
    The optional \meta{namespace} is only a placeholder in order to prevent any conflicts between duplicate \meta{key}s.
    \DescribeMacro{\loadpayload} The same behaviour counts for \cmd{\loadpayload}\oarg{namespace}\marg{filename}.

    \clearpage
    \section{Parameter Specification}
    Every parameter specified has a \meta{type} set.
    Optionally there is a choice between setting a \meta{default} or a \meta{placeholder} for the parameter.
    \begin{description}
        \item[bool] Next to the textual representation of \textit{true} and \textit{false}, it provides a \LaTeX{} command using the \texttt{ifthen} package.
        Therefore, only the \meta{default} setting makes sense.\\[5pt]
        \hspace*{5pt}\parbox{\linewidth-5pt}{%
            \hfill\href{example/example-specification.yaml}{\texttt{Parameters}}\hfill\hspace*{\columnsep}%
            \hfill\href{example/example.yaml}{\texttt{Variables}}\hfill\hspace*{\columnsep}}\\%
        \showexample{1}{1-3}{1}{1-1}
        \DescribeMacro{\param}
        With a boolean type the \cmd{\param}\oarg{namespace}\marg{name} returns either \textit{true} or \textit{false}.
        \DescribeMacro{\ifparam}
        Additionally, it provides the \cmd{\ifparam}\marg{name}\marg{true code}\marg{false code} command for top level boolean types.
        The macro is just a wrapper for the boolean package \texttt{ifthen}, which supports spaces in names.
        \item[string] representing a piece of text.
        All \TeX{} related symbols in the text, like \textbackslash, \% or \#, are escaped.\\
        \showexample{4}{4-6}{2}{2-2}
        \DescribeMacro{\param} A string type can easily be placed in \LaTeX{} using the \cmd{\param} command.
        \item[number] representing a number, like the number type of Lua.
        In most cases it's necessary to use \meta{default} instead of \meta{placeholder}, especially when the number is used in calculations, since a placeholder will cause errors in \LaTeX{} calculations.\\
        \showexample{7}{7-9}{3}{3-3}
        A number type can also be used with \cmd{\param}, just like the string type.
        \item[list] representing a list of values.
        The value type is specified by \meta{value type}.
        A \meta{default} setting can be set.
        Due to its structure, a \meta{placeholder} would be somewhat incompatible with the corresponding macros.
        However, a placeholder can be simulated by setting the placeholders as children of the \meta{default} list, as demonstrated in the example.\\
        \showexample{10}{10-15}{4}{4-6}
        \DescribeMacro{\param}
        Command \cmd{\param} concatenates every item with command \cmd{\paramlist@conjunction}.
        \DescribeMacro{\paramlist@conjunction}
        By default, the conjunction is set to `\texttt{,\textasciitilde}'.

        \DescribeMacro{\forparamlist}
        There's also the \cmd{\forparamlist}\oarg{namespace}\marg{name}\marg{csname} command, which takes an additional \meta{csname} and will execute it for every item in the list.
        This command doesn't handle advanced features like altering the conjunction.
        Though, some utility commands will be set, which are only available in the \meta{csname}s implementation, in order to achieve the same goal.
        \item[object] representing a list of key value pairs.
        This parameter type requires a \meta{fields} specification to be set.
        Any field must be of type \texttt{bool}, \texttt{string} or \texttt{number}.\\
        \showexample{16}{16-24}{7}{7-9}
        There is no support for the \cmd{\param} command.
        \DescribeMacro{\paramfield}
        In order to show to contents there is the \cmd{\paramfield}\marg{name}\marg{field} command.

        \item[table] representing a table.
        This parameter type requires a \meta{columns} specification to be set.\\
        \showexample[20pt]{25}{25}{10}{10}
    \end{description}

%    \section{Full Specification Example}
%    \lstinputlisting[language=yaml,frame=single,caption={example-specification.yaml},captionpos=t,numbers=left]{example/example-specification.yaml}
%
%    \lstinputlisting[language=yaml,frame=single,caption={exampe.yaml},captionpos=t,numbers=left]{example/example.yaml}
\end{document}
