\documentclass{article}
\usepackage{gitinfo-lua}
\usepackage{elpi}
\usepackage{listings}

\loadrecipe[example-specification.yaml]

\setlength{\parindent}{0pt}

\begin{document}
    \title{ELPI Example\thanks{This example corresponds to \texttt{elpi} version \gitversion{} written on \gitdate.}}
    \author{\dogitauthors[\\]}
    \maketitle

    \begin{minipage}[t]{.5\linewidth}
        \section*{Before values loaded}

        Boolean example:
        \ifparam{bool example}{TRUE}{FALSE}\\

        String example:
        \param{string example}\\

        Number example:
        \param{number example}\\

        List example:\\
        \parbox{.7 \linewidth}{\param{list example}}\\
        \begin{enumerate}
            \newcommand\formatitem[1]{\item #1}
            \forparamlist{list example}{formatitem}
        \end{enumerate}
    \end{minipage}%
    \begin{minipage}[t]{.5\linewidth}
        \section*{After values loaded}
        \loadpayload[example.yaml]

        Boolean example:
        \ifparam{bool example}{TRUE}{FALSE}\\

        String example:
        \param{string example}\\

        Number example:
        \param{number example}\\

        List example:\\
        using \verb|\param{list example}|:\\
        \parbox{.7 \linewidth}{\param{list example}}\\

        using \verb|\forparamlist{list example}|:\\[-1em]
        \begin{enumerate}
            \newcommand\formatitem[1]{\item #1}
            \forparamlist{list example}{formatitem}
        \end{enumerate}

        Object example:\\

    \end{minipage}

    \section*{Specification File}
    \lstinputlisting[language=YAML,numbers=left,xleftmargin=15pt,caption={example-specification.yaml}]{example-specification.yaml}


    \section*{Payload File}
    \lstinputlisting[language=YAML,numbers=left,xleftmargin={15pt},caption={example.yaml}]{example.yaml}
\end{document}
